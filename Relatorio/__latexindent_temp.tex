\documentclass[12pt, a4paper]{article}

\usepackage{amsmath}
\usepackage[utf8]{inputenc}
\usepackage[brazil]{babel}
\usepackage{microtype}
\usepackage{indentfirst}
\usepackage{graphicx}
\usepackage[style=abnt, backend=biber]{biblatex}
\usepackage{csquotes}
\addbibresource{Referencias.bib}
\usepackage{times}

\title{Relatório ITP}
\author{Thiago Oliveira Coelho}

\begin{document}

\maketitle

\tableofcontents

\clearpage

\section{Introdução}

Com o advento da globalização, as barreiras tradicionais ao comércio internacional, como as tarifárias, têm se tornado menores o que diminui as oportunidades para implementação de medidas protecionistas.\cite{maskus2000quantifying}. Isso tem causado aparecimento de diversas barreiras não tarifárias (BNTs), que impedem o fluxo internacional de bens.
Apesar de estas barreiras poderem ser legítimas, por exemplo para corrigir eventuais externalidades negativas advindas do produto importado,  o fato é que estas terão impacto nas importações do país. Este impacto pode ser positivo ou negativo, dependendo do setor analisado. Em geral, normas de importação tendem a diminuir o comércio para bens primários e impulsionar o comércio de bens mais complexos \cite{moenius}.

\section{Metodologia}

Considerando os trabalhos que visam estabelecer quantitativamente o impacto das notificações, será utilizado um modelo de gravidade cujos estimadores serão estabelecidos por PPML (Poisson Pseudo Maximum Likelihood). 

\subsection{Modelos de gravidade}

Os modelos de gravidade são utilizados majoritariamente desde a década de 60 para a explicação de fluxos de comércio internacional. Originalmente derivado do modelo de Newton, utilizava a distância entre os dois objetos (países) e a massa deles (PIB), para explicar tal fluxo. Com o tempo, o desenvolvimento da área de economia internacional têm tornado o modelo cada vez mais teóricamente embasado e representativo da realidade. 

\section{Dados e fontes}

Bases de dados utilizadas:

\begin{enumerate}
    \item Notificações: https://www.epingalert.org/en;
    \item Valor de exportações: https://comtrade.un.org/;
    \item PIB : Banco mundial, \cite{WB};
    \item Distanciamento: \cite{CEPII} 
\end{enumerate}

\printbibliography

\end{document}