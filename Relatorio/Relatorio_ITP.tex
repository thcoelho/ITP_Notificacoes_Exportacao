\documentclass[12pt, a4paper]{article}

\usepackage{amsmath}
\usepackage[utf8]{inputenc}
\usepackage[brazil]{babel}
\usepackage{microtype}
\usepackage{indentfirst}
\usepackage{graphicx}
\usepackage{booktabs}
\usepackage[style=abnt, backend=biber]{biblatex}
\usepackage{csquotes}
\addbibresource{Referencias.bib}
\usepackage{times}

\title{Relatório ITP}
\author{Thiago Oliveira Coelho}

\begin{document}

\maketitle

\tableofcontents

\clearpage

\section{Introdução}

Com o advento da globalização, as barreiras tradicionais ao comércio internacional, como as tarifárias, têm se tornado menores o que diminui as oportunidades para implementação de medidas protecionistas.\cite{maskus2000quantifying}. Isso tem causado aparecimento de diversas barreiras não tarifárias (BNTs), que impedem o fluxo internacional de bens.
Apesar de estas barreiras poderem ser legítimas, por exemplo para corrigir eventuais externalidades negativas advindas do produto importado,  o fato é que estas terão impacto nas importações do país. Este impacto pode ser positivo ou negativo, dependendo do setor analisado. Em geral, normas de importação tendem a diminuir o comércio para bens primários e impulsionar o comércio de bens mais complexos \cite{moenius}.

\section{Metodologia}

Considerando os trabalhos que visam estabelecer quantitativamente o impacto das notificações, será utilizado um modelo de gravidade cujos estimadores serão estabelecidos por PPML (Poisson Pseudo Maximum Likelihood). 

\subsection{Modelos de gravidade}

Os modelos de gravidade são utilizados majoritariamente desde a década de 60 para a explicação de fluxos de comércio internacional. Originalmente derivado do modelo de Newton, utilizava a distância entre os dois objetos (países) e a massa deles (PIB), para explicar tal fluxo. Com o tempo, o desenvolvimento da área de economia internacional têm tornado o modelo cada vez mais teóricamente embasado e representativo da realidade. 

\subsection{Obtenção de estimadores}

Dada a característica de haver grandes quantidades de fluxos de troca com valor zero, o estimador utilizado será o de \emph{Poisson Pseudo Maximum Likelihood} (PPML). Isso se deve pelo fato de tal metodologia se portar melhor dados muitos valores nulos. Tal método de estimação também gera resultados consistentes na presença de heterocedasticidade. 

\subsection{Variáveis do modelo}

\begin{equation}
    \ln X_{JT} = \ln YB_t + \ln YE_{JT} + Dist_J + \ln TRF_{JT} 
\end{equation}

Onde:

\begin{itemize}
    \item $X_{JT}$ = Valor de exportação do Brasil para o país $J$ no período $T$;
    \item $YB_T$ = Renda do Brasil no período $t$;
    \item $YE_{JT}$ = Renda do país J no período $t$;
    \item $Dist_J$ = Distância entre o Brasil e o país $J$;
    \item $TRF_{JT}$ = Valor da tarifa efetivamente aplicada pelo país $J$ ao Brasil no período $T$.
\end{itemize}

Obs: Todas aquelas variáveis que não são dummies estão sendo transformadas por meio de logaritmo natural, assim como pede a especificação de \cite{Log_Of_Gravity}.


\section{Dados e fontes}

Bases de dados utilizadas:

\begin{enumerate}
    \item Notificações: https://www.epingalert.org/en;
    \item Valor de exportações: https://comtrade.un.org/;
    \item PIB : Banco mundial, \cite{WB};
    \item Distanciamento: \cite{CEPII};
    \item Tarifas: WITS
\end{enumerate}

\section{Resultados}

\subsection{Regressão Simples}
\begin{center}
\begin{tabular}{lclc}
\toprule
\textbf{Dep. Variable:}                                            &   ln\_trade   & \textbf{  No. Iterations:    } &     5       \\
\textbf{Model:}                                                    &      GLM      & \textbf{  Df Residuals:      } &    7442     \\
\textbf{Model Family:}                                             &    Poisson    & \textbf{  Df Model:          } &     20      \\
\textbf{Link Function:}                                            &      log      & \textbf{  Scale:             } &    1.0000   \\
\textbf{Method:}                                                   &      IRLS     & \textbf{  Log-Likelihood:    } &   -18927.   \\
\textbf{Covariance Type:}                                          &      HC1      & \textbf{  Deviance:          } &    4944.0   \\
\bottomrule
\end{tabular}
\begin{tabular}{lcccccc}
                                                                   & \textbf{coef} & \textbf{std err} & \textbf{t} & \textbf{P$> |$t$|$} & \textbf{[0.025} & \textbf{0.975]}  \\
\midrule
\textbf{Animal.health}                                             &       0.0046  &        0.001     &     4.152  &         0.000        &        0.002    &        0.007     \\
\textbf{Consumer.information}                                      &      -0.0106  &        0.012     &    -0.893  &         0.372        &       -0.034    &        0.013     \\
\textbf{Food.safety}                                               &      -0.0053  &        0.001     &    -9.394  &         0.000        &       -0.006    &       -0.004     \\
\textbf{Harmonization}                                             &      -0.5198  &        0.062     &    -8.348  &         0.000        &       -0.642    &       -0.398     \\
\textbf{Lower.barriers.to.trade}                                   &      -0.0333  &        0.008     &    -4.433  &         0.000        &       -0.048    &       -0.019     \\
\textbf{Other}                                                     &      -0.0234  &        0.019     &    -1.219  &         0.223        &       -0.061    &        0.014     \\
\textbf{Plant.protection}                                          &       0.0006  &        0.001     &     0.489  &         0.625        &       -0.002    &        0.003     \\
\textbf{Prevention.of.deceptive.practices.and.consumer.protection} &      -0.0525  &        0.006     &    -9.286  &         0.000        &       -0.064    &       -0.041     \\
\textbf{Protect.humans.from.animal.plant.pest.or.disease}          &       0.0028  &        0.002     &     1.782  &         0.075        &       -0.000    &        0.006     \\
\textbf{Protect.territory.from.other.damage.from.pests}            &      -0.0368  &        0.008     &    -4.801  &         0.000        &       -0.052    &       -0.022     \\
\textbf{Protection.of.Human.health.or.Safety}                      &      -0.0074  &        0.002     &    -4.613  &         0.000        &       -0.011    &       -0.004     \\
\textbf{Protection.of.animal.or.plant.life.or.health}              &      -0.0847  &        0.040     &    -2.143  &         0.032        &       -0.162    &       -0.007     \\
\textbf{Protection.of.the.environment}                             &       0.0268  &        0.005     &     5.071  &         0.000        &        0.016    &        0.037     \\
\textbf{Quality.requirements}                                      &      -0.0142  &        0.009     &    -1.525  &         0.127        &       -0.032    &        0.004     \\
\textbf{ln\_gdp\_d}                                                &       0.0384  &        0.001     &    26.368  &         0.000        &        0.036    &        0.041     \\
\textbf{ln\_gdp\_o}                                                &       0.0306  &        0.015     &     2.004  &         0.045        &        0.001    &        0.060     \\
\textbf{comrelig}                                                  &      -0.0039  &        0.010     &    -0.399  &         0.690        &       -0.023    &        0.015     \\
\textbf{gatt\_d}                                                   &      -0.1327  &        0.019     &    -7.140  &         0.000        &       -0.169    &       -0.096     \\
\textbf{gatt\_o}                                                   &       0.8822  &        0.427     &     2.066  &         0.039        &        0.045    &        1.719     \\
\textbf{eu\_d}                                                     &      -0.4352  &        0.037     &   -11.867  &         0.000        &       -0.507    &       -0.363     \\
\textbf{distw}                                                     &   -1.569e-06  &      5.8e-07     &    -2.706  &         0.007        &    -2.71e-06    &    -4.32e-07     \\
\bottomrule
\end{tabular}
%\caption{Generalized Linear Model Regression Results}
\end{center}

\printbibliography

\end{document}