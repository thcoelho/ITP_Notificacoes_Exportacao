\documentclass[12pt, a4paper]{article}

\usepackage{amsmath}
\usepackage[utf8]{inputenc}
\usepackage[brazil]{babel}
\usepackage{microtype}
\usepackage{indentfirst}
\usepackage{graphicx}
\usepackage{booktabs}
\usepackage[style=abnt, backend=biber]{biblatex}
\usepackage{csquotes}
\addbibresource{Referencias.bib}
\usepackage{times}

\title{Relatório ITP}
\author{Thiago Oliveira Coelho}

\begin{document}

\maketitle

\tableofcontents

\clearpage

\section{Introdução}

\subsection{As barreiras não tarifárias ao comércio}
Com o advento da globalização o valor das tarifas internacionais tem caído ao longo do tempo, isto diminui as oportunidades para implementação de medidas protecionistas \cite{maskus2000quantifying}. O foco passa então a ser as barreiras não tarifárias (BNTs), conjunto de fatores não tarifários que impedem o fluxo de bens internacionais. Dentre as barreiras não tarifárias são inclusas as normas as quais os produtores devem se conformar para exportar para determinado país. Tais normas devem ser comunicadas aos demais países por meio de notificações, e são regulamentadas pela Organização Mundial do Comércio (OMC). Elas podem ser sanitárias ou fitossanitárias (SPS), associadas a produtos alimentícios, ou barreiras técnicas ao comércio (TBT), que ditam regras para que o produto se adeque ao mercado interno.

Apesar de tais barreiras poderem ser legítimas, por exemplo para corrigir eventuais externalidades negativas advindas do produto importado,  o fato é que estas terão impacto no comércio do país. Este impacto pode ser positivo ou negativo, dependendo do setor afetado, do objetivo da notificação e de seu conteúdo. Os acordos SPS e TBT possuem o objetivo de:

\begin{itemize}
    \item Encorajar membros a adequar seus produtos com base em regulamentações internacionais, visando melhor facilidade ao comércio internacional;
    \item Impedir a criação de barreiras arbitrárias e que não possuam embasamento científico;
    \item Manter a soberania do país para especificar normas de acordo com sua situação específica, desde que estas não visem prejudicar o fluxo de bens internacional;
    \item Exigir que novas normas que afetem o comércio sejam notificadas aos demais países, e que "pontos focais" devem ser estabelecidos. Estes são escritórios que visam melhorar a  transparência das normas estabelecidas pelo país em questão. O ponto focal brasileiro é o Inmetro.
\end{itemize}

O presente trabalho visa explorar os impactos das notificações relacionadas a normas sanitárias e fitossanitárias  (SPS) e barreiras técnicas  (TBT). As notificações são categorizadas de acordo com seu objetivo, e é a partir de tal objetivo que distinguiremos os diferentes tipos de notificações e seus respectivos impactos.

\subsection{A relação das barreiras técnicas com o setor agroexportador}

Levando em conta a estrtura de exportações do brasil:

\begin{figure}[h]
    \centering
    \includegraphics[scale=0.75]{Imagens/exportacoes.png}
    \caption{Exportações Brasileiras por setor em 2018. Fonte: WITS}
\end{figure}

Vemos que a maior parte das exportações brasileiras estão relacionadas ao setor vegetal (19\%), também há representação significativa do setor animal (6\%). Pela importância de tais categorias, é importante identificarmos eventuais gargalos a sua comercialização, entre estes as barreiras não tarifárias. Para observarmos  tais impactos iremos separar as notificações emitidas em categorias diferentes dependendo de seus objetivos e analisar o impacto de suas emissões sobre o valor de exportação do Brasil no mesmo período. Em geral a literatura  nos diz que normas de importação tendem a diminuir o comércio para bens primários e impulsionar o comércio de bens mais complexos \cite{moenius}. Regulamentações particularmente informativas irão diminuir a incerteza em tais transações, porém acompanhar tais notificações pode ser difícil para o micro e pequeno participante do agronegócio. 

\section{Metodologia}

Serão utilizados modelos econométricos de gravidade, que por meio de cálculos Poisson Pseudo Maximum Likelihood (PPML), irão retornar coeficientes para cada variável do modelo. Assim podemos observar os efeitos que cada variável posusi sobre os fluxos exportadores brasileiros.

\subsection{Modelos de gravidade}

Os modelos de gravidade são utilizados majoritariamente desde a década de 60 para a explicação de fluxos de comércio internacional. Originalmente derivado do modelo de Newton, utilizava a distância entre os dois objetos (países) e a massa deles (PIB), para explicar tal fluxo. Com o tempo, o desenvolvimento da área de economia internacional têm tornado o modelo cada vez mais teóricamente embasado e representativo da realidade, incluindo diversas variáveis como população, tarifas e efeitos fixos para unidade temporal e país \cite{nascimento2013evoluccao}.

\subsection{Estimador PPML}
Considerando os trabalhos que visam estabelecer quantitativamente o impacto das notificações (\cite{impactosprodquimicos} e \cite{ALMEIDA2014}), será utilizado um modelo de gravidade cujos estimadores serão estabelecidos por PPML (Poisson Pseudo Maximum Likelihood). Tal modo de estimação permite menor viés ao trabalharmos com fluxos internacionais, visto que muitas das observações de emissão de notificações ou de tarifas podem ser de valor nulo. Um modelo de regressão estimado por mínimos quadrados ordinários não está preparado para lidar com tais fluxos zero, o que deixa o resultado enviesado \cite{Log_Of_Gravity}.

\subsection{Software}
Foi utilizado o pacote GME, escrito na linguagem de programação Python pela USITC (United States Internacional Trade Commission). Tal pacote roda diagnósticos da ppml para confirmar que não houve erros na estimação, além de permitir a fácil inclusão dos efeitos fixos e da estimação setor por setor.

\subsection{Variáveis do modelo}

\begin{equation}
    \ln X_{D} = 
    \ln GDP_{O} 
    + \ln GDP_{D} 
    + \ln dist 
    + \ln Tariff 
    + \sum_{i=1}^{n} P 
    % + Eu_{D} 
    + contig 
    % + comrelig
    % + gatt_o 
    % + gatt_D
\end{equation}

Onde:

\begin{itemize}
    \item $X$ = Valor de exportação do Brasil para o país de destino no período;
    \item $GDP_{O}$ = Produto interno bruto nominal do Brasil no período;
    \item $GDP_{D}$ = Produto interno bruto nominal do país de destino no período;
    \item $dist$ = Distância em Quilômetros  do Brasil para o país de destino;
    \item $Tariff$ = Média ponderada das tarifas efetivamente aplicadas pelo país de destino ao Brasil no período;
    \item $\sum_{i=1}^{n} P$ = Conjunto de variáveis que indicam a quantidade de notificações emitidas por cada país e objetivo;
    % \item $Eu_{D}$ = Dummy que indica se o país de destino faz parte da União Europeia.
    \item $contig$ = Dummy que indica se o Brasil possui fronteiras em comum com o país de destino;
    % \item $comrelig$ = Dummy que indica se os países possuem religiões difundidas em comum;
    % \item $gatt_O$ = Se o país (origem) é membro da Organização mundial do comércio;
    % \item $gatt_D$ = Se o país (destino) é membro da Organização mundial do comércio.
\end{itemize}

Esta equação nos indica que as alterações percentuais no valor de exportações brasileiras são funções do PIB do Brasil e do país importador, da distância entre eles e da emissão ou não de notificações naquela substrato de tempo. O período utilizado na análise é anual.

\subsection{Dados e fontes}

Bases de dados utilizadas:

\begin{enumerate}
    \item Notificações: Sistema de alerta de notificações EPING: https://www.epingalert.org/en;
    \item Valor de exportações: Base de dados estatísticos sobre comércio internacional das Nações Unidas: https://comtrade.un.org/;
    \item PIB: Sistema de dados abertos do banco mundial: https://data.worldbank.org/;
    \item Distância e fronteiras em comum: Base de dados para modelos gravitacionais da Centre d'Etudes Prospectives et d'Informations Internationales: \cite{CEPII};
    \item Tarifas: WITS (World Integrated Trade Solution): https://wits.worldbank.org/.
\end{enumerate}

\subsection{Interpretação}

\subsubsection{Coeficientes}

Os coeficientes representam impactos percentuais nas exportações dada alteração de 1 ponto percentual na variável explicativa. Isso se dá pelo fato das variáveis estarem logaritmadas, o que não é verdadeiro para as variáveis relacionadas a notificações. Isto ocorre pelo fato do logaritmo natural de 0 ser indefinido, e como muitos objetivos de notificações possuem valor 0 para certos anos, logaritmar tal variável prejudicaria a acurácia do modelo. O valor da alteração causado pelas notificações portanto é de difícil interpretação, e é recomendável que o valor de tais coeficientes seja utilizado para determinar somente se o impacto de tal tipo de notificação é positivo ou negativo.

\subsubsection{P-Valor}
O p valor é uma medida estatística simples que nos indica a relevância de uma variável aos diversos níveis de significância. Nos dá uma probabilidade de encontrarmos uma amostra mais representativa da realidade do que a que temos atualmente. Se o p-valor de uma variável for menor que 5\% (0,05) então podemos concluir que tal variável rejeita a hipótese nula, que nos diz que a variável independente não imapcta a variável dependente (exportações).


\newpage
\section{Resultados}

\subsection{Regressão com efeitos fixos e setores agregados}

% Please add the following required packages to your document preamble:
% \usepackage{booktabs}
\begin{table}[h]
    \begin{tabular}{@{}lll@{}}
    \toprule
    \textbf{Variável}                                                                       & \textbf{Coeficiente}         & \textbf{P-Valor} \\ \midrule
    \multicolumn{1}{l|}{\textbf{Animal.health}}                                             & \multicolumn{1}{l|}{-0.0008} & 0.508           \\ \midrule
    \multicolumn{1}{l|}{\textbf{Consumer.information}}                                      & \multicolumn{1}{l|}{0.0158}  & 0.048            \\ \midrule
    \multicolumn{1}{l|}{\textbf{Food.safety}}                                               & \multicolumn{1}{l|}{-0.0002} & 0.850           \\ \midrule
    \multicolumn{1}{l|}{\textbf{Harmonization}}                                             & \multicolumn{1}{l|}{2.1376}  & 0.000           \\ \midrule
    \multicolumn{1}{l|}{\textbf{Lower.barriers.to.trade}}                                   & \multicolumn{1}{l|}{-0.0680} & 0.211           \\ \midrule
    \multicolumn{1}{l|}{\textbf{Other}}                                                     & \multicolumn{1}{l|}{-0.0510} & 0.061           \\ \midrule
    \multicolumn{1}{l|}{\textbf{Plant.protection}}                                          & \multicolumn{1}{l|}{-0.0041} & 0.000           \\ \midrule
    \multicolumn{1}{l|}{\textbf{Prevention.of.deceptive.practices.and.consumer.protection}} & \multicolumn{1}{l|}{-0.0018} & 0.460           \\ \midrule
    \multicolumn{1}{l|}{\textbf{Protect.humans.from.animal.plant.pest.or.disease}}          & \multicolumn{1}{l|}{0.0008}  & 0.743            \\ \midrule
    \multicolumn{1}{l|}{\textbf{Protect.territory.from.other.damage.from.pests}}            & \multicolumn{1}{l|}{-0.0159} & 0.000           \\ \midrule
    \multicolumn{1}{l|}{\textbf{Protection.of.Human.health.or.Safety}}                      & \multicolumn{1}{l|}{-0.0018} & 0.126           \\ \midrule
    \multicolumn{1}{l|}{\textbf{Protection.of.animal.or.plant.life.or.health}}              & \multicolumn{1}{l|}{-0.0747} & 0.008           \\ \midrule
    \multicolumn{1}{l|}{\textbf{Protection.of.the.environment}}                             & \multicolumn{1}{l|}{-0.5433} & 0.000         \\ \midrule
    \multicolumn{1}{l|}{\textbf{Quality.requirements}}                                      & \multicolumn{1}{l|}{-0.0117} & 0.055           \\ \midrule
    \multicolumn{1}{l|}{\textbf{ln\_gdp\_d}}                                                & \multicolumn{1}{l|}{0.0438}  & 0.000            \\ \midrule
    \multicolumn{1}{l|}{\textbf{ln\_gdp\_o}}                                                & \multicolumn{1}{l|}{0.1635}  & 0.000            \\ \midrule
    \multicolumn{1}{l|}{\textbf{ln\_dist}}                                                  & \multicolumn{1}{l|}{-0.3453} & 0.000           \\ \midrule
    \multicolumn{1}{l|}{\textbf{contig}}                                                    & \multicolumn{1}{l|}{-0.1671} & 0.053           \\ \midrule
    \textbf{ln\_Tariff}                                                                     & 0.0084                       & 0.096            \\ \bottomrule
    \end{tabular}
    \caption{Regressão com todos os setores}
    \label{tab:my-table}
\end{table}

Por representar uma gama de notificações que não se encaixam nos demais objetivos, a interpretação da variável outros é difícil. Junto com as variáveis de informação ao consumidor e harmonização, possui impacto positivo sobre as exportações brasileiras. Isso condiz com a literatura, que afirma que as notificações informativas ao exportador beneficiam o comércio por comunicar aspectos importantes do mercado. O impacto da variável harmonização é maior, o que faz sentido, visto que as notificações com tal objetivo visam a adequação das regras de comércio em bases internacionais, equalizando as normas e deixando o ambiente mais competitivo. 

Podemos perceber que as variáveis relacionadas a proteção são as que impactam negativamente o comércio internacional. Isso pode significar que a adaptação dos produtos a esta pode ser particularmente difícil ou cara, o que seria compreensível, visto que tenderiam a ser mais rigorosas.


% \subsection{Regressão sem efeitos fixos e resultados setor por setor}


% \subsubsection{Setor 01 - Animais vivos}

\begin{table}[ht]
    \begin{center}
    \begin{tabular}{lcccccc}
                                                              & \textbf{Coeficiente} &\textbf{P-Valor} \\
    \midrule
    \textbf{Animal.health}                                    &       0.0076  &        0.767        \\
    \textbf{Food.safety}                                      &       0.0606  &        0.044        \\
    \textbf{Other}                                            &       0.1392  &        0.068        \\
    \textbf{Plant.protection}                                 &       0.0370  &        0.066        \\
    \textbf{Protect.humans.from.animal.plant.pest.or.disease} &      -0.1003  &        0.034        \\
    \textbf{Protect.territory.from.other.damage.from.pests}   &       0.1847  &        0.000        \\
    \textbf{ln\_gdp\_d}                                       &      -0.0059  &        0.562        \\
    \textbf{ln\_gdp\_o}                                       &       0.1672  &        0.309        \\
    \textbf{comrelig}                                         &      -0.2677  &        0.026        \\
    \textbf{gatt\_d}                                          &      -1.4241  &        0.769        \\
    \textbf{ln\_dist}                                         &      -0.0546  &        0.459        \\
    \textbf{contig}                                           &       0.1301  &        0.073        \\
    \textbf{ln\_Tariff}                                       &      -0.0375  &        0.119        \\
    \bottomrule
    \end{tabular}
    \caption{Resultados Regressão Sem Efeitos Fixos - Setor 1}
    \end{center}
    \end{table} 

\subsubsection{Setor 02 - Carnes e miudezas, comestíveis}

\begin{center}
\begin{tabular}{lcccccc}
                                                          & \textbf{Coeficiente} & \textbf{Erro padrão} &\textbf{P$> |$t$|$}\\
\midrule
\textbf{Animal.health}                                    &      -0.0168  &        0.015     &        0.279 \\
\textbf{Food.safety}                                      &       0.0096  &        0.010     &        0.357 \\
\textbf{Plant.protection}                                 &       0.0352  &        0.068     &        0.604 \\
\textbf{Protect.humans.from.animal.plant.pest.or.disease} &      -0.0148  &        0.019     &        0.433 \\
\textbf{Protect.territory.from.other.damage.from.pests}   &       0.2203  &        0.105     &        0.036 \\
\textbf{ln\_gdp\_d}                                       &       0.6314  &        0.315     &        0.045 \\
\textbf{ln\_gdp\_o}                                       &      76.8483  &       42.841     &        0.073 \\
\textbf{comrelig}                                         &      27.6711  &       14.041     &        0.049 \\
\textbf{gatt\_d}                                          &   -2192.8017  &     1220.545     &        0.072 \\
\textbf{ln\_dist}                                         &      -0.7920  &        0.555     &        0.153 \\
\textbf{contig}                                           &       0.1221  &        0.109     &        0.262 \\
\textbf{ln\_Tariff}                                       &       0.0143  &        0.022     &        0.521 \\
\bottomrule
\end{tabular}
%\caption{Generalized Linear Model Regression Results}
\end{center} 
\subsubsection{Setor 03 - Peixes e crustáceos, moluscos e outros invertebrados aquáticos}

\begin{center}
\begin{tabular}{lcccccc}
                                                                   & \textbf{Coeficiente} & \textbf{Erro Padrão} &\textbf{P$> |$t$|$}\\
\midrule
\textbf{Animal.health}                                             &      -0.0181  &        0.024     &        0.455\\
\textbf{Food.safety}                                               &       0.0236  &        0.044     &        0.590\\
\textbf{Plant.protection}                                          &       0.0823  &        0.117     &        0.483\\
\textbf{Prevention.of.deceptive.practices.and.consumer.protection} &       0.7658  &        0.123     &        0.000\\
\textbf{Protect.humans.from.animal.plant.pest.or.disease}          &       0.0149  &        0.025     &        0.554\\
\textbf{Protect.territory.from.other.damage.from.pests}            &       0.1217  &        0.212     &        0.566\\
\textbf{Protection.of.Human.health.or.Safety}                      &      -0.0221  &        0.017     &        0.197\\
\textbf{ln\_gdp\_d}                                                &      -0.0687  &        0.033     &        0.040\\
\textbf{ln\_gdp\_o}                                                &      99.9094  &       96.000     &        0.298\\
\textbf{comrelig}                                                  &       0.3401  &        0.164     &        0.039\\
\textbf{gatt\_d}                                                   &   -2840.8862  &     2730.589     &        0.298\\
\textbf{ln\_dist}                                                  &       0.3936  &        0.125     &        0.002\\
\textbf{contig}                                                    &       0.4713  &        0.288     &        0.102\\
\textbf{ln\_Tariff}                                                &      -0.6192  &        0.116     &        0.000\\
\bottomrule
\end{tabular}
%\caption{Generalized Linear Model Regression Results}
\end{center} 

\subsubsection{Setor 04 - Leite e lacticínios; ovos de aves; mel natural; produtos comestíveis de origem animal, não especificados nem compreendidos em outros capítulos}

\begin{center}
\begin{tabular}{lclc}
\toprule
\textbf{Dep. Variable:}                                            &   ln\_trade   & \textbf{  No. Iterations:    } &     4       \\
\textbf{Model:}                                                    &      GLM      & \textbf{  Df Residuals:      } &     18      \\
\textbf{Model Family:}                                             &    Poisson    & \textbf{  Df Model:          } &     52      \\
\textbf{Link Function:}                                            &      log      & \textbf{  Scale:             } &    1.0000   \\
\textbf{Method:}                                                   &      IRLS     & \textbf{  Log-Likelihood:    } &   -159.96   \\
\textbf{Covariance Type:}                                          &      HC1      & \textbf{  Deviance:          } &    6.5363   \\
\bottomrule
\end{tabular}
\begin{tabular}{lcccccc}
                                                                   & \textbf{coef} & \textbf{std err} & \textbf{t} & \textbf{P$> |$t$|$} & \textbf{[0.025} & \textbf{0.975]}  \\
\midrule
\textbf{Animal.health}                                             &       0.0461  &        0.034     &     1.353  &         0.176        &       -0.025    &        0.118     \\
\textbf{Consumer.information}                                      &       0.4312  &        0.321     &     1.345  &         0.179        &       -0.242    &        1.105     \\
\textbf{Food.safety}                                               &      -0.0261  &        0.014     &    -1.857  &         0.063        &       -0.056    &        0.003     \\
\textbf{Lower.barriers.to.trade}                                   &      -0.4286  &        0.190     &    -2.253  &         0.024        &       -0.828    &       -0.029     \\
\textbf{Plant.protection}                                          &       0.0704  &        0.056     &     1.267  &         0.205        &       -0.046    &        0.187     \\
\textbf{Prevention.of.deceptive.practices.and.consumer.protection} &      -0.1515  &        0.120     &    -1.267  &         0.205        &       -0.403    &        0.100     \\
\textbf{Protect.humans.from.animal.plant.pest.or.disease}          &       0.0420  &        0.016     &     2.634  &         0.008        &        0.008    &        0.075     \\
\textbf{Protect.territory.from.other.damage.from.pests}            &      -0.0520  &        0.088     &    -0.594  &         0.553        &       -0.236    &        0.132     \\
\textbf{Protection.of.Human.health.or.Safety}                      &       0.0278  &        0.030     &     0.929  &         0.353        &       -0.035    &        0.091     \\
\textbf{ln\_gdp\_d}                                                &       0.5046  &        0.263     &     1.917  &         0.055        &       -0.049    &        1.058     \\
\textbf{ln\_gdp\_o}                                                &      79.5572  &       36.898     &     2.156  &         0.031        &        2.038    &      157.076     \\
\textbf{comrelig}                                                  &      -6.2107  &        2.860     &    -2.171  &         0.030        &      -12.220    &       -0.201     \\
\textbf{gatt\_d}                                                   &   -2263.2433  &     1053.515     &    -2.148  &         0.032        &    -4476.595    &      -49.891     \\
\textbf{ln\_dist}                                                  &      -1.1385  &        0.780     &    -1.459  &         0.145        &       -2.778    &        0.501     \\
\textbf{contig}                                                    &      -0.1598  &        0.130     &    -1.231  &         0.218        &       -0.433    &        0.113     \\
\textbf{ln\_Tariff}                                                &      -0.0158  &        0.018     &    -0.878  &         0.380        &       -0.053    &        0.022     \\
\textbf{Year\_fe\_2007}                                            &      39.7265  &       18.425     &     2.156  &         0.031        &        1.017    &       78.435     \\
\textbf{Year\_fe\_2008}                                            &      24.5133  &       11.350     &     2.160  &         0.031        &        0.668    &       48.358     \\
\textbf{Year\_fe\_2009}                                            &      26.2262  &       12.122     &     2.164  &         0.030        &        0.759    &       51.693     \\
\textbf{Year\_fe\_2010}                                            &       3.9653  &        1.777     &     2.232  &         0.026        &        0.232    &        7.699     \\
\textbf{Year\_fe\_2011}                                            &      -7.7527  &        3.586     &    -2.162  &         0.031        &      -15.287    &       -0.219     \\
\textbf{Country.Code\_fe\_ALB}                                     &       1.6105  &        1.034     &     1.557  &         0.119        &       -0.562    &        3.783     \\
\textbf{Country.Code\_fe\_ARE}                                     &       0.6367  &        0.235     &     2.709  &         0.007        &        0.143    &        1.130     \\
\textbf{Country.Code\_fe\_AUS}                                     &       1.4663  &        0.697     &     2.103  &         0.035        &        0.002    &        2.931     \\
\textbf{Country.Code\_fe\_CAN}                                     &       1.8444  &        1.000     &     1.844  &         0.065        &       -0.256    &        3.945     \\
\textbf{Country.Code\_fe\_CHL}                                     &       3.6213  &        2.510     &     1.443  &         0.149        &       -1.652    &        8.894     \\
\textbf{Country.Code\_fe\_CHN}                                     &      -0.6354  &        0.749     &    -0.848  &         0.396        &       -2.210    &        0.939     \\
\textbf{Country.Code\_fe\_COL}                                     &       4.6122  &        2.707     &     1.704  &         0.088        &       -1.076    &       10.300     \\
\textbf{Country.Code\_fe\_CPV}                                     &       7.2774  &        4.044     &     1.800  &         0.072        &       -1.218    &       15.773     \\
\textbf{Country.Code\_fe\_CRI}                                     &       4.8780  &        3.095     &     1.576  &         0.115        &       -1.624    &       11.380     \\
\textbf{Country.Code\_fe\_DOM}                                     &       5.4123  &        3.132     &     1.728  &         0.084        &       -1.168    &       11.993     \\
\textbf{Country.Code\_fe\_EEC}                                     &       4.9258  &        3.107     &     1.586  &         0.113        &       -1.601    &       11.452     \\
\textbf{Country.Code\_fe\_FJI}                                     &       3.6945  &        1.646     &     2.245  &         0.025        &        0.237    &        7.152     \\
\textbf{Country.Code\_fe\_GTM}                                     &       4.9474  &        3.081     &     1.606  &         0.108        &       -1.526    &       11.421     \\
\textbf{Country.Code\_fe\_HND}                                     &       6.2729  &        3.402     &     1.844  &         0.065        &       -0.875    &       13.421     \\
\textbf{Country.Code\_fe\_ISL}                                     &       1.7453  &        1.185     &     1.473  &         0.141        &       -0.744    &        4.235     \\
\textbf{Country.Code\_fe\_JPN}                                     &      -0.3186  &        0.755     &    -0.422  &         0.673        &       -1.904    &        1.267     \\
\textbf{Country.Code\_fe\_KOR}                                     &       0.5010  &        0.471     &     1.064  &         0.287        &       -0.488    &        1.490     \\
\textbf{Country.Code\_fe\_LKA}                                     &       2.1728  &        0.920     &     2.361  &         0.018        &        0.239    &        4.106     \\
\textbf{Country.Code\_fe\_MAR}                                     &       0.4736  &        0.664     &     0.713  &         0.476        &       -0.922    &        1.869     \\
\textbf{Country.Code\_fe\_MKD}                                     &       1.9722  &        1.167     &     1.689  &         0.091        &       -0.481    &        4.425     \\
\textbf{Country.Code\_fe\_MLI}                                     &       0.8008  &        1.335     &     0.600  &         0.549        &       -2.005    &        3.607     \\
\textbf{Country.Code\_fe\_NIC}                                     &       6.7124  &        3.543     &     1.894  &         0.058        &       -0.732    &       14.156     \\
\textbf{Country.Code\_fe\_NPL}                                     &       2.2525  &        1.038     &     2.171  &         0.030        &        0.073    &        4.432     \\
\textbf{Country.Code\_fe\_NZL}                                     &       1.9325  &        0.947     &     2.041  &         0.041        &       -0.057    &        3.922     \\
\textbf{Country.Code\_fe\_OMN}                                     &       1.2774  &        0.637     &     2.006  &         0.045        &       -0.061    &        2.615     \\
\textbf{Country.Code\_fe\_PER}                                     &       4.9765  &        2.928     &     1.700  &         0.089        &       -1.175    &       11.128     \\
\textbf{Country.Code\_fe\_PHL}                                     &       5.9220  &        2.483     &     2.385  &         0.017        &        0.705    &       11.139     \\
\textbf{Country.Code\_fe\_RUS}                                     &      -0.7227  &        0.276     &    -2.621  &         0.009        &       -1.302    &       -0.143     \\
\textbf{Country.Code\_fe\_SAU}                                     &       0.1959  &        0.101     &     1.934  &         0.053        &       -0.017    &        0.409     \\
\textbf{Country.Code\_fe\_SLV}                                     &       6.5156  &        3.358     &     1.940  &         0.052        &       -0.540    &       13.571     \\
\textbf{Country.Code\_fe\_SWZ}                                     &       2.9703  &        1.723     &     1.724  &         0.085        &       -0.650    &        6.590     \\
\textbf{Country.Code\_fe\_THA}                                     &       0.6701  &        0.415     &     1.614  &         0.106        &       -0.202    &        1.542     \\
\bottomrule
\end{tabular}
%\caption{Generalized Linear Model Regression Results}
\end{center} 
\subsubsection{Setor 05 - Outros produtos de origem animal, não especificados nem compreendidos em outros capítulos}

\begin{center}
\begin{tabular}{lcccccc}
                                                          & \textbf{Coeficiente} & \textbf{Erro padrão} &\textbf{P$> |$t$|$}\\
\midrule
\textbf{Animal.health}                                    &       0.0244  &        0.012     &        0.045\\
\textbf{Food.safety}                                      &       0.0490  &        0.024     &        0.038\\
\textbf{Protect.humans.from.animal.plant.pest.or.disease} &      -0.0230  &        0.016     &        0.150\\
\textbf{ln\_gdp\_d}                                       &       0.0115  &        0.019     &        0.546\\
\textbf{ln\_gdp\_o}                                       &       0.3758  &        0.220     &        0.087\\
\textbf{comrelig}                                         &      -0.2275  &        0.111     &        0.041\\
\textbf{gatt\_d}                                          &      -8.1391  &        6.245     &        0.192\\
\textbf{ln\_dist}                                         &      -0.0238  &        0.077     &        0.759\\
\bottomrule
\end{tabular}
%\caption{Generalized Linear Model Regression Results}
\end{center} 

\subsubsection{Setor 06 - Plantas vivas e produtos de floricultura}


\begin{table}
    
    \begin{center}
        \begin{tabular}{lcccccc}
            & \textbf{coef} & \textbf{P$> |$t$|$}\\
            \midrule
\textbf{Animal.health}                                    &      -0.1445 &         0.049\\
\textbf{Food.safety}                                      &      -0.0128 &         0.711\\
\textbf{Plant.protection}                                 &      -0.0009 &         0.454\\
\textbf{Protect.humans.from.animal.plant.pest.or.disease} &       0.0318 &         0.219\\
\textbf{Protect.territory.from.other.damage.from.pests}   &      -0.0225 &         0.252\\
\textbf{ln\_gdp\_d}                                       &       0.0039 &         0.675\\
\textbf{ln\_gdp\_o}                                       &       0.1266 &         0.231\\
\textbf{comrelig}                                         &      -0.1426 &         0.102\\
\textbf{gatt\_d}                                          &      -0.0937 &         0.118\\
\textbf{gatt\_o}                                          &      -0.5283 &         0.856\\
\textbf{ln\_dist}                                         &      -0.0420 &         0.441\\
\textbf{contig}                                           &      -0.1005 &         0.223\\
\textbf{ln\_Tariff}                                       &      -0.0955 &         0.000\\
\bottomrule
\end{tabular}
\caption{Resultados Regressão Sem Efeitos Fixos - Setor 6}
\end{center}
\end{table} 
\subsubsection{Setor 07 - Produtos hortícolas, plantas, raízes e tubérculos, comestíveis}

\begin{center}
\begin{tabular}{lclc}
\toprule
\textbf{Dep. Variable:}                                            &   ln\_trade   & \textbf{  No. Iterations:    } &     4       \\
\textbf{Model:}                                                    &      GLM      & \textbf{  Df Residuals:      } &     12      \\
\textbf{Model Family:}                                             &    Poisson    & \textbf{  Df Model:          } &     45      \\
\textbf{Link Function:}                                            &      log      & \textbf{  Scale:             } &    1.0000   \\
\textbf{Method:}                                                   &      IRLS     & \textbf{  Log-Likelihood:    } &   -131.46   \\
\textbf{Covariance Type:}                                          &      HC1      & \textbf{  Deviance:          } &    15.781   \\
\bottomrule
\end{tabular}
\begin{tabular}{lcccccc}
                                                                   & \textbf{coef} & \textbf{std err} & \textbf{t} & \textbf{P$> |$t$|$} & \textbf{[0.025} & \textbf{0.975]}  \\
\midrule
\textbf{Animal.health}                                             &       0.4148  &        0.170     &     2.437  &         0.015        &        0.044    &        0.786     \\
\textbf{Food.safety}                                               &      -0.3881  &        0.076     &    -5.104  &         0.000        &       -0.554    &       -0.222     \\
\textbf{Lower.barriers.to.trade}                                   &       0.0325  &        0.308     &     0.106  &         0.916        &       -0.638    &        0.703     \\
\textbf{Plant.protection}                                          &       0.0971  &        0.062     &     1.567  &         0.117        &       -0.038    &        0.232     \\
\textbf{Prevention.of.deceptive.practices.and.consumer.protection} &       1.5740  &        0.283     &     5.559  &         0.000        &        0.957    &        2.191     \\
\textbf{Protect.humans.from.animal.plant.pest.or.disease}          &       0.2326  &        0.059     &     3.950  &         0.000        &        0.104    &        0.361     \\
\textbf{Protect.territory.from.other.damage.from.pests}            &      -0.3913  &        0.126     &    -3.109  &         0.002        &       -0.666    &       -0.117     \\
\textbf{ln\_gdp\_d}                                                &      -0.1692  &        0.088     &    -1.922  &         0.055        &       -0.361    &        0.023     \\
\textbf{ln\_gdp\_o}                                                &     119.7595  &      145.997     &     0.820  &         0.412        &     -198.341    &      437.860     \\
\textbf{comrelig}                                                  &      -0.4137  &        0.338     &    -1.223  &         0.221        &       -1.151    &        0.323     \\
\textbf{gatt\_d}                                                   &       0.3223  &        0.303     &     1.064  &         0.287        &       -0.338    &        0.982     \\
\textbf{gatt\_o}                                                   &   -3405.3946  &     4152.331     &    -0.820  &         0.412        &    -1.25e+04    &     5641.758     \\
\textbf{ln\_dist}                                                  &       0.6900  &        0.177     &     3.891  &         0.000        &        0.304    &        1.076     \\
\textbf{contig}                                                    &       0.1269  &        0.310     &     0.409  &         0.682        &       -0.549    &        0.802     \\
\textbf{ln\_Tariff}                                                &       0.0825  &        0.037     &     2.232  &         0.026        &        0.002    &        0.163     \\
\textbf{Year\_fe\_2007}                                            &      59.5747  &       72.594     &     0.821  &         0.412        &      -98.594    &      217.743     \\
\textbf{Year\_fe\_2008}                                            &      36.2460  &       44.783     &     0.809  &         0.418        &      -61.328    &      133.820     \\
\textbf{Year\_fe\_2009}                                            &      37.9742  &       47.685     &     0.796  &         0.426        &      -65.922    &      141.871     \\
\textbf{Year\_fe\_2010}                                            &       5.3410  &        6.905     &     0.774  &         0.439        &       -9.703    &       20.385     \\
\textbf{Year\_fe\_2011}                                            &     -11.5299  &       14.188     &    -0.813  &         0.416        &      -42.442    &       19.383     \\
\textbf{Country.Code\_fe\_ALB}                                     &       0.1322  &        0.346     &     0.382  &         0.703        &       -0.622    &        0.887     \\
\textbf{Country.Code\_fe\_ARE}                                     &      -0.2084  &        0.090     &    -2.320  &         0.020        &       -0.404    &       -0.013     \\
\textbf{Country.Code\_fe\_AUS}                                     &      -0.4393  &        0.338     &    -1.300  &         0.194        &       -1.176    &        0.297     \\
\textbf{Country.Code\_fe\_BHR}                                     &      -0.3612  &        0.351     &    -1.029  &         0.303        &       -1.126    &        0.403     \\
\textbf{Country.Code\_fe\_BRB}                                     &       1.0809  &        0.399     &     2.706  &         0.007        &        0.211    &        1.951     \\
\textbf{Country.Code\_fe\_BRN}                                     &      -0.9771  &        0.325     &    -3.003  &         0.003        &       -1.686    &       -0.268     \\
\textbf{Country.Code\_fe\_CAN}                                     &       0.2149  &        0.365     &     0.589  &         0.556        &       -0.581    &        1.011     \\
\textbf{Country.Code\_fe\_CHL}                                     &       0.5731  &        0.263     &     2.178  &         0.029        &       -0.000    &        1.146     \\
\textbf{Country.Code\_fe\_CHN}                                     &      -0.1937  &        0.298     &    -0.651  &         0.515        &       -0.842    &        0.455     \\
\textbf{Country.Code\_fe\_COL}                                     &       0.0733  &        0.227     &     0.323  &         0.747        &       -0.421    &        0.567     \\
\textbf{Country.Code\_fe\_CPV}                                     &      -0.7949  &        0.582     &    -1.366  &         0.172        &       -2.063    &        0.473     \\
\textbf{Country.Code\_fe\_CRI}                                     &      -0.1610  &        0.249     &    -0.648  &         0.517        &       -0.703    &        0.381     \\
\textbf{Country.Code\_fe\_ECU}                                     &       0.9575  &        0.184     &     5.201  &         0.000        &        0.556    &        1.359     \\
\textbf{Country.Code\_fe\_EGY}                                     &       0.2600  &        0.185     &     1.403  &         0.161        &       -0.144    &        0.664     \\
\textbf{Country.Code\_fe\_FJI}                                     &      -0.0071  &        0.569     &    -0.013  &         0.990        &       -1.247    &        1.233     \\
\textbf{Country.Code\_fe\_IDN}                                     &      -0.1482  &        0.219     &    -0.677  &         0.498        &       -0.625    &        0.329     \\
\textbf{Country.Code\_fe\_IND}                                     &       0.0151  &        0.235     &     0.064  &         0.949        &       -0.496    &        0.526     \\
\textbf{Country.Code\_fe\_JOR}                                     &      -0.5786  &        0.312     &    -1.855  &         0.064        &       -1.258    &        0.101     \\
\textbf{Country.Code\_fe\_KOR}                                     &      -0.7640  &        0.244     &    -3.136  &         0.002        &       -1.295    &       -0.233     \\
\textbf{Country.Code\_fe\_MAR}                                     &      -0.3075  &        0.249     &    -1.235  &         0.217        &       -0.850    &        0.235     \\
\textbf{Country.Code\_fe\_MEX}                                     &      -0.0423  &        0.322     &    -0.132  &         0.895        &       -0.743    &        0.659     \\
\textbf{Country.Code\_fe\_MLI}                                     &      -0.1985  &        0.512     &    -0.388  &         0.698        &       -1.314    &        0.917     \\
\textbf{Country.Code\_fe\_NIC}                                     &       0.2990  &        0.158     &     1.890  &         0.059        &       -0.046    &        0.644     \\
\textbf{Country.Code\_fe\_NPL}                                     &      -1.1356  &        0.359     &    -3.167  &         0.002        &       -1.917    &       -0.354     \\
\textbf{Country.Code\_fe\_NZL}                                     &       0.5084  &        0.187     &     2.724  &         0.006        &        0.102    &        0.915     \\
\textbf{Country.Code\_fe\_OMN}                                     &      -1.1301  &        0.271     &    -4.169  &         0.000        &       -1.721    &       -0.539     \\
\bottomrule
\end{tabular}
%\caption{Generalized Linear Model Regression Results}
\end{center} 

\subsubsection{Setor 08 - Frutas; cascas de citrinos e de melões}

\begin{center}
\begin{tabular}{lcccccc}
                                                                   & \textbf{Coeficiente} & \textbf{Erro padrão} & \textbf{P$> |$t$|$}\\
\midrule
\textbf{Animal.health}                                             &      -0.0695  &        0.053     &         0.187\\
\textbf{Food.safety}                                               &       0.0844  &        0.066     &         0.200\\
\textbf{Plant.protection}                                          &      -0.0046  &        0.027     &         0.867\\
\textbf{Prevention.of.deceptive.practices.and.consumer.protection} &      -0.4114  &        0.128     &         0.001\\
\textbf{Protect.humans.from.animal.plant.pest.or.disease}          &       0.0411  &        0.027     &         0.122\\
\textbf{Protect.territory.from.other.damage.from.pests}            &       0.0283  &        0.037     &         0.440\\
\textbf{ln\_gdp\_d}                                                &      -0.0077  &        0.028     &         0.783\\
\textbf{ln\_gdp\_o}                                                &    -199.5235  &       68.139     &         0.003\\
\textbf{comrelig}                                                  &      -0.4189  &        0.072     &         0.000\\
\textbf{gatt\_d}                                                   &      -0.2041  &        0.097     &         0.035\\
\textbf{gatt\_o}                                                   &    5674.2754  &     1938.088     &         0.003\\
\textbf{ln\_dist}                                                  &       0.3599  &        0.070     &         0.000\\
\textbf{contig}                                                    &       0.3006  &        0.106     &         0.004\\
\textbf{ln\_Tariff}                                                &       0.1776  &        0.033     &         0.000\\
\bottomrule
\end{tabular}
%\caption{Generalized Linear Model Regression Results}
\end{center} 
\subsubsection{Setor 09 - Café, chá, mate e especiarias}

\begin{table}[ht]
    \begin{center}
        \begin{tabular}{lcccccc}
            & \textbf{Coeficiente} & \textbf{P-Valor}\\
            \midrule
\textbf{Animal.health}                                             &       0.1078  &         0.010   \\
\textbf{Food.safety}                                               &       0.0033  &         0.615   \\
\textbf{Lower.barriers.to.trade}                                   &       0.0439  &         0.075   \\
\textbf{Plant.protection}                                          &       0.0143  &         0.478   \\
\textbf{Prevention.of.deceptive.practices.and.consumer.protection} &      -4.6513  &         0.047   \\
\textbf{Protect.humans.from.animal.plant.pest.or.disease}          &      -0.0014  &         0.742   \\
\textbf{Protect.territory.from.other.damage.from.pests}            &       0.0454  &         0.211   \\
\textbf{Protection.of.Human.health.or.Safety}                      &       1.5500  &         0.049   \\
\textbf{ln\_gdp\_d}                                                &       0.0620  &         0.000   \\
\textbf{ln\_gdp\_o}                                                &      -0.1237  &         0.137   \\
\textbf{comrelig}                                                  &      -0.0830  &         0.021   \\
\textbf{gatt\_d}                                                   &       4.6583  &         0.050   \\
\bottomrule
\end{tabular}
\caption{Resultados Regressão Sem Efeitos Fixos - Setor 9}
\end{center}
\end{table} 

\newpage

\subsubsection{Setor 10 - Cereais}

\begin{center}
\begin{tabular}{lclc}
\toprule
\textbf{Dep. Variable:}                                            &   ln\_trade   & \textbf{  No. Iterations:    } &     4       \\
\textbf{Model:}                                                    &      GLM      & \textbf{  Df Residuals:      } &    640      \\
\textbf{Model Family:}                                             &    Poisson    & \textbf{  Df Model:          } &     69      \\
\textbf{Link Function:}                                            &      log      & \textbf{  Scale:             } &    1.0000   \\
\textbf{Method:}                                                   &      IRLS     & \textbf{  Log-Likelihood:    } &   -1715.2   \\
\textbf{Covariance Type:}                                          &      HC1      & \textbf{  Deviance:          } &    320.52   \\
\bottomrule
\end{tabular}
\begin{tabular}{lcccccc}
                                                                   & \textbf{coef} & \textbf{std err} & \textbf{t} & \textbf{P$> |$t$|$} & \textbf{[0.025} & \textbf{0.975]}  \\
\midrule
\textbf{Animal.health}                                             &      -0.0011  &        0.003     &    -0.335  &         0.738        &       -0.008    &        0.006     \\
\textbf{Consumer.information}                                      &       0.3054  &        0.105     &     2.900  &         0.004        &        0.099    &        0.512     \\
\textbf{Food.safety}                                               &      -0.0054  &        0.004     &    -1.311  &         0.190        &       -0.014    &        0.003     \\
\textbf{Lower.barriers.to.trade}                                   &      -0.0167  &        0.038     &    -0.441  &         0.659        &       -0.091    &        0.058     \\
\textbf{Plant.protection}                                          &       0.0075  &        0.007     &     1.040  &         0.299        &       -0.007    &        0.022     \\
\textbf{Prevention.of.deceptive.practices.and.consumer.protection} &       0.1521  &        0.047     &     3.269  &         0.001        &        0.061    &        0.243     \\
\textbf{Protect.humans.from.animal.plant.pest.or.disease}          &       0.0049  &        0.006     &     0.772  &         0.440        &       -0.008    &        0.017     \\
\textbf{Protect.territory.from.other.damage.from.pests}            &      -0.0094  &        0.016     &    -0.585  &         0.558        &       -0.041    &        0.022     \\
\textbf{Protection.of.Human.health.or.Safety}                      &      -0.0978  &        0.030     &    -3.222  &         0.001        &       -0.157    &       -0.038     \\
\textbf{Protection.of.animal.or.plant.life.or.health}              &      -0.2954  &        0.058     &    -5.062  &         0.000        &       -0.410    &       -0.181     \\
\textbf{Quality.requirements}                                      &      -0.0069  &        0.033     &    -0.209  &         0.834        &       -0.072    &        0.058     \\
\textbf{ln\_gdp\_d}                                                &       0.0023  &        0.091     &     0.025  &         0.980        &       -0.176    &        0.180     \\
\textbf{ln\_gdp\_o}                                                &      75.7156  &       21.900     &     3.457  &         0.001        &       32.710    &      118.721     \\
\textbf{comrelig}                                                  &      65.6997  &       23.215     &     2.830  &         0.005        &       20.114    &      111.286     \\
\textbf{gatt\_d}                                                   &   -2280.8208  &      639.428     &    -3.567  &         0.000        &    -3536.450    &    -1025.191     \\
\textbf{ln\_dist}                                                  &      12.4213  &        4.718     &     2.633  &         0.008        &        3.157    &       21.686     \\
\textbf{contig}                                                    &       0.0425  &        0.027     &     1.556  &         0.120        &       -0.011    &        0.096     \\
\textbf{ln\_Tariff}                                                &       0.0372  &        0.036     &     1.020  &         0.308        &       -0.034    &        0.109     \\
\textbf{Year\_fe\_2007}                                            &      37.5853  &       10.888     &     3.452  &         0.001        &       16.204    &       58.967     \\
\textbf{Year\_fe\_2008}                                            &      23.2029  &        6.720     &     3.453  &         0.001        &       10.006    &       36.400     \\
\textbf{Year\_fe\_2009}                                            &      24.7513  &        7.164     &     3.455  &         0.001        &       10.683    &       38.819     \\
\textbf{Year\_fe\_2010}                                            &       3.6679  &        1.046     &     3.507  &         0.000        &        1.614    &        5.721     \\
\textbf{Year\_fe\_2011}                                            &      -7.3766  &        2.121     &    -3.478  &         0.001        &      -11.541    &       -3.212     \\
\textbf{Country.Code\_fe\_ALB}                                     &      16.3005  &        5.198     &     3.136  &         0.002        &        6.093    &       26.508     \\
\textbf{Country.Code\_fe\_ARE}                                     &      13.0944  &        4.293     &     3.050  &         0.002        &        4.664    &       21.525     \\
\textbf{Country.Code\_fe\_ARG}                                     &     -19.5359  &        6.731     &    -2.902  &         0.004        &      -32.754    &       -6.318     \\
\textbf{Country.Code\_fe\_ARM}                                     &      14.2083  &        4.339     &     3.274  &         0.001        &        5.687    &       22.729     \\
\textbf{Country.Code\_fe\_AUS}                                     &      -6.4053  &        2.556     &    -2.506  &         0.012        &      -11.425    &       -1.386     \\
\textbf{Country.Code\_fe\_BLZ}                                     &     -17.3217  &        6.800     &    -2.547  &         0.011        &      -30.674    &       -3.969     \\
\textbf{Country.Code\_fe\_CAN}                                     &     -10.0716  &        3.658     &    -2.753  &         0.006        &      -17.255    &       -2.888     \\
\textbf{Country.Code\_fe\_CHL}                                     &     -17.8938  &        6.219     &    -2.877  &         0.004        &      -30.106    &       -5.682     \\
\textbf{Country.Code\_fe\_CHN}                                     &       8.7787  &        2.875     &     3.054  &         0.002        &        3.134    &       14.423     \\
\textbf{Country.Code\_fe\_COL}                                     &     -29.7929  &       10.545     &    -2.825  &         0.005        &      -50.499    &       -9.086     \\
\textbf{Country.Code\_fe\_CRI}                                     &     -29.6210  &       10.576     &    -2.801  &         0.005        &      -50.388    &       -8.854     \\
\textbf{Country.Code\_fe\_DOM}                                     &     -31.5801  &       11.383     &    -2.774  &         0.006        &      -53.932    &       -9.228     \\
\textbf{Country.Code\_fe\_ECU}                                     &     -29.8192  &       10.592     &    -2.815  &         0.005        &      -50.619    &       -9.020     \\
\textbf{Country.Code\_fe\_EEC}                                     &     -29.6587  &       10.590     &    -2.801  &         0.005        &      -50.455    &       -8.863     \\
\textbf{Country.Code\_fe\_EGY}                                     &      15.5494  &        5.201     &     2.989  &         0.003        &        5.336    &       25.763     \\
\textbf{Country.Code\_fe\_GTM}                                     &     -33.2218  &       11.971     &    -2.775  &         0.006        &      -56.729    &       -9.715     \\
\textbf{Country.Code\_fe\_HKG}                                     &       3.0768  &        0.812     &     3.789  &         0.000        &        1.482    &        4.671     \\
\textbf{Country.Code\_fe\_HND}                                     &     -33.3390  &       12.212     &    -2.730  &         0.006        &      -57.319    &       -9.359     \\
\textbf{Country.Code\_fe\_IDN}                                     &       7.9140  &        2.438     &     3.246  &         0.001        &        3.126    &       12.702     \\
\textbf{Country.Code\_fe\_IND}                                     &      10.5402  &        3.475     &     3.033  &         0.002        &        3.717    &       17.363     \\
\textbf{Country.Code\_fe\_ISR}                                     &      14.6195  &        4.871     &     3.001  &         0.003        &        5.054    &       24.185     \\
\textbf{Country.Code\_fe\_JOR}                                     &      14.0876  &        4.487     &     3.140  &         0.002        &        5.277    &       22.898     \\
\textbf{Country.Code\_fe\_JPN}                                     &       7.9170  &        2.525     &     3.136  &         0.002        &        2.959    &       12.875     \\
\textbf{Country.Code\_fe\_KEN}                                     &       0.7914  &        0.210     &     3.763  &         0.000        &        0.378    &        1.204     \\
\textbf{Country.Code\_fe\_KOR}                                     &       5.7308  &        1.673     &     3.425  &         0.001        &        2.445    &        9.016     \\
\textbf{Country.Code\_fe\_KWT}                                     &      12.7751  &        4.092     &     3.122  &         0.002        &        4.741    &       20.810     \\
\textbf{Country.Code\_fe\_MAC}                                     &       3.3310  &        0.454     &     7.332  &         0.000        &        2.439    &        4.223     \\
\textbf{Country.Code\_fe\_MAR}                                     &      19.6647  &        6.656     &     2.955  &         0.003        &        6.595    &       32.734     \\
\textbf{Country.Code\_fe\_MEX}                                     &     -35.7613  &       12.745     &    -2.806  &         0.005        &      -60.789    &      -10.734     \\
\textbf{Country.Code\_fe\_MKD}                                     &      15.7796  &        4.998     &     3.157  &         0.002        &        5.965    &       25.594     \\
\textbf{Country.Code\_fe\_MLI}                                     &      22.6791  &        7.573     &     2.995  &         0.003        &        7.807    &       37.551     \\
\textbf{Country.Code\_fe\_NIC}                                     &     -32.1547  &       11.814     &    -2.722  &         0.006        &      -55.354    &       -8.955     \\
\textbf{Country.Code\_fe\_NOR}                                     &      12.8866  &        4.246     &     3.035  &         0.002        &        4.548    &       21.225     \\
\textbf{Country.Code\_fe\_NPL}                                     &      10.5654  &        3.084     &     3.426  &         0.001        &        4.510    &       16.620     \\
\textbf{Country.Code\_fe\_NZL}                                     &       0.9809  &        0.182     &     5.393  &         0.000        &        0.624    &        1.338     \\
\textbf{Country.Code\_fe\_OMN}                                     &      13.1202  &        4.130     &     3.177  &         0.001        &        5.010    &       21.230     \\
\textbf{Country.Code\_fe\_PER}                                     &     -26.5694  &        9.467     &    -2.806  &         0.005        &      -45.160    &       -7.978     \\
\textbf{Country.Code\_fe\_PHL}                                     &     -40.4106  &       14.869     &    -2.718  &         0.007        &      -69.608    &      -11.214     \\
\textbf{Country.Code\_fe\_PRY}                                     &     -17.2245  &        6.090     &    -2.828  &         0.005        &      -29.183    &       -5.266     \\
\textbf{Country.Code\_fe\_QAT}                                     &      13.1441  &        4.187     &     3.140  &         0.002        &        4.923    &       21.365     \\
\textbf{Country.Code\_fe\_RUS}                                     &      12.5044  &        4.216     &     2.966  &         0.003        &        4.226    &       20.783     \\
\textbf{Country.Code\_fe\_SAU}                                     &      14.8020  &        4.897     &     3.023  &         0.003        &        5.187    &       24.417     \\
\textbf{Country.Code\_fe\_SGP}                                     &       7.0480  &        1.947     &     3.620  &         0.000        &        3.225    &       10.871     \\
\textbf{Country.Code\_fe\_SLV}                                     &     -33.6315  &       12.305     &    -2.733  &         0.006        &      -57.795    &       -9.468     \\
\textbf{Country.Code\_fe\_SWZ}                                     &      11.5857  &        3.438     &     3.369  &         0.001        &        4.834    &       18.338     \\
\textbf{Country.Code\_fe\_THA}                                     &       9.1028  &        2.819     &     3.229  &         0.001        &        3.568    &       14.638     \\
\textbf{Country.Code\_fe\_UKR}                                     &      14.2183  &        4.793     &     2.967  &         0.003        &        4.807    &       23.630     \\
\bottomrule
\end{tabular}
%\caption{Generalized Linear Model Regression Results}
\end{center}

\newpage

\subsubsection{Setor 11 - Produtos da indústria de moagem; malte; amidos e féculas; inulina; glúten de trigo}

\begin{center}
\begin{tabular}{lcccccc}
                                                                   & \textbf{Coeficiente} & \textbf{Erro padrão} &\textbf{P$> |$t$|$}\\
\midrule
\textbf{Animal.health}                                             &      -3.2418  &        0.039     &        0.000\\
\textbf{Food.safety}                                               &       1.8531  &        0.021     &        0.000\\
\textbf{Plant.protection}                                          &       0.1714  &        0.032     &        0.000\\
\textbf{Prevention.of.deceptive.practices.and.consumer.protection} &       0.0657  &        0.012     &        0.000\\
\textbf{Protect.humans.from.animal.plant.pest.or.disease}          &      -0.2587  &        0.009     &        0.000\\
\textbf{Protect.territory.from.other.damage.from.pests}            &       2.4013  &        0.048     &        0.000\\
\textbf{Protection.of.Human.health.or.Safety}                      &       0.0004  &        0.001     &        0.397\\
\textbf{Quality.requirements}                                      &       0.4108  &        0.028     &        0.000\\
\textbf{ln\_gdp\_d}                                                &      -1.5803  &        0.029     &        0.000\\
\textbf{ln\_gdp\_o}                                                &     321.1951  &       11.695     &        0.000\\
\textbf{comrelig}                                                  &      19.5737  &        0.253     &        0.000\\
\textbf{gatt\_d}                                                   &   -8759.6755  &      334.600     &        0.000\\
\textbf{ln\_dist}                                                  &     -35.3856  &        0.479     &        0.000\\
\textbf{contig}                                                    &     -58.1114  &        0.796     &        0.000\\
\textbf{ln\_Tariff}                                                &      -1.0134  &        0.007     &        0.000\\
\bottomrule
\end{tabular}
%\caption{Generalized Linear Model Regression Results}
\end{center}

\newpage

\subsubsection{Setor 12 - Sementes e frutos oleaginosos; grãos, sementes e frutos diversos; plantas industriais ou medicinais; palhas e forragens}

\begin{center}
\begin{tabular}{lcccccc}
                                                                   & \textbf{Coeficiente} & \textbf{Erro padrão} &\textbf{P$> |$t$|$}\\
\midrule
\textbf{Animal.health}                                             &       0.0053  &        0.012     &        0.652\\
\textbf{Consumer.information}                                      &       0.7894  &        0.115     &        0.000\\
\textbf{Food.safety}                                               &       0.0035  &        0.007     &        0.619\\
\textbf{Harmonization}                                             &       1.0998  &        0.194     &        0.000\\
\textbf{Lower.barriers.to.trade}                                   &      -0.2151  &        0.040     &        0.000\\
\textbf{Plant.protection}                                          &      -0.0188  &        0.008     &        0.023\\
\textbf{Prevention.of.deceptive.practices.and.consumer.protection} &      -0.0152  &        0.014     &        0.275\\
\textbf{Protect.humans.from.animal.plant.pest.or.disease}          &      -0.0158  &        0.018     &        0.373\\
\textbf{Protect.territory.from.other.damage.from.pests}            &      -0.0082  &        0.024     &        0.732\\
\textbf{Protection.of.Human.health.or.Safety}                      &       0.0110  &        0.006     &        0.091\\
\textbf{Protection.of.animal.or.plant.life.or.health}              &      -0.1823  &        0.059     &        0.002\\
\textbf{Protection.of.the.environment}                             &      -0.3207  &        0.046     &        0.000\\
\textbf{ln\_gdp\_d}                                                &      -0.3519  &        0.137     &        0.010\\
\textbf{ln\_gdp\_o}                                                &     113.3384  &       28.499     &        0.000\\
\textbf{comrelig}                                                  &       6.5232  &        2.386     &        0.006\\
\textbf{gatt\_d}                                                   &   -3192.8712  &      810.468     &        0.000\\
\textbf{ln\_dist}                                                  &      -2.0886  &        0.374     &        0.000\\
\textbf{contig}                                                    &       0.0177  &        0.039     &        0.651\\
\textbf{ln\_Tariff}                                                &       0.0354  &        0.041     &        0.382\\
\bottomrule
\end{tabular}
%\caption{Generalized Linear Model Regression Results}
\end{center}

\newpage

\subsubsection{Setor 13 - Gomas, resinas e outros sucos e extractos vegetais}

\begin{table}[ht]
    \begin{center}
        \begin{tabular}{lcccccc}
                                                                   & \textbf{coef} & \textbf{P$> |$t$|$}\\
\midrule
\textbf{Animal.health}                                             &       0.1349  &         0.422       \\
\textbf{Food.safety}                                               &       0.0076  &         0.868       \\
\textbf{Plant.protection}                                          &      -0.4128  &         0.041       \\
\textbf{Prevention.of.deceptive.practices.and.consumer.protection} &       0.0573  &         0.458       \\
\textbf{Protect.humans.from.animal.plant.pest.or.disease}          &      -0.0306  &         0.663       \\
\textbf{Protect.territory.from.other.damage.from.pests}            &      -0.3081  &         0.626       \\
\textbf{Protection.of.Human.health.or.Safety}                      &      -0.0824  &         0.000       \\
\textbf{ln\_gdp\_d}                                                &       0.0406  &         0.004       \\
\textbf{ln\_gdp\_o}                                                &      -0.0237  &         0.956       \\
\textbf{comrelig}                                                  &      -0.6690  &         0.004       \\
\textbf{gatt\_d}                                                   &       5.8972  &         0.637       \\
\textbf{ln\_dist}                                                  &      -0.3724  &         0.010       \\
\textbf{contig}                                                    &      -0.0022  &         0.984       \\
\textbf{ln\_Tariff}                                                &       0.2325  &         0.069       \\
\bottomrule
\end{tabular}
\caption{Resultados Regressão Sem Efeitos Fixos - Setor 13}
\end{center}
\end{table}

\subsubsection{Setor 14 - Matérias para entrançar e outros produtos de origem vegetal, não especificados nem compreendidos noutros capítulos}

\begin{center}
\begin{tabular}{lcccccc}
                                                          & \textbf{Coeficiente} & \textbf{Erro padrão} &\textbf{P$> |$t$|$}\\
\midrule
\textbf{Animal.health}                                    &       0.1570  &        0.211     &        0.456\\
\textbf{Food.safety}                                      &       0.0332  &        0.139     &        0.811\\
\textbf{Plant.protection}                                 &      -0.0520  &        0.080     &        0.518\\
\textbf{Protect.humans.from.animal.plant.pest.or.disease} &       0.0533  &        0.093     &        0.564\\
\textbf{Protect.territory.from.other.damage.from.pests}   &      -0.6114  &        0.151     &        0.000\\
\textbf{ln\_gdp\_d}                                       &       0.0269  &        0.018     &        0.130\\
\textbf{ln\_gdp\_o}                                       &       0.0685  &        0.015     &        0.000\\
\bottomrule
\end{tabular}
%\caption{Generalized Linear Model Regression Results}
\end{center}

\newpage

\subsubsection{Setor 15 - Gorduras e óleos animais ou vegetais; produtos da sua dissociação; gorduras alimentares elaboradas; ceras de origem animal ou vegetal}

\begin{center}
\begin{tabular}{lcccccc}
                                                                   & \textbf{Coeficiente} & \textbf{Erro padrão} &\textbf{P$> |$t$|$}\\
\midrule
\textbf{Animal.health}                                             &       0.0124  &        0.013     &        0.340\\
\textbf{Consumer.information}                                      &       0.0943  &        0.073     &        0.198\\
\textbf{Food.safety}                                               &       0.0067  &        0.009     &        0.449\\
\textbf{Lower.barriers.to.trade}                                   &       0.0516  &        0.075     &        0.490\\
\textbf{Plant.protection}                                          &      -0.0515  &        0.065     &        0.430\\
\textbf{Prevention.of.deceptive.practices.and.consumer.protection} &       0.0483  &        0.029     &        0.091\\
\textbf{Protect.humans.from.animal.plant.pest.or.disease}          &      -0.0421  &        0.023     &        0.068\\
\textbf{Protect.territory.from.other.damage.from.pests}            &      -0.0091  &        0.051     &        0.859\\
\textbf{Protection.of.Human.health.or.Safety}                      &       0.0038  &        0.010     &        0.700\\
\textbf{Quality.requirements}                                      &       2.8619  &        0.962     &        0.003\\
\textbf{ln\_gdp\_d}                                                &      -0.5357  &        0.251     &        0.033\\
\textbf{ln\_gdp\_o}                                                &     236.0764  &       21.634     &        0.000\\
\textbf{comrelig}                                                  &      -2.6279  &        1.609     &        0.102\\
\textbf{gatt\_d}                                                   &   -6664.7941  &      621.275     &        0.000\\
\textbf{ln\_dist}                                                  &      -3.6243  &        1.161     &        0.002\\
\textbf{contig}                                                    &       0.1241  &        0.091     &        0.172\\
\textbf{ln\_Tariff}                                                &       0.0809  &        0.032     &        0.012\\
\bottomrule
\end{tabular}
%\caption{Generalized Linear Model Regression Results}
\end{center} 


\printbibliography


\end{document}