\documentclass[12pt, a4paper]{article}

\usepackage{amsmath}
\usepackage[utf8]{inputenc}
\usepackage[brazil]{babel}
\usepackage{microtype}
\usepackage{indentfirst}
\usepackage{graphicx}
\usepackage{booktabs}
\usepackage[style=abnt, backend=biber]{biblatex}
\usepackage{csquotes}
\addbibresource{Referencias.bib}
\usepackage{times}

\title{Relatório ITP}
\author{Thiago Oliveira Coelho}

\begin{document}

\maketitle

\tableofcontents

\clearpage

\section{Introdução}

Com o advento da globalização, as barreiras tradicionais ao comércio internacional, como as tarifárias, têm se tornado menores o que diminui as oportunidades para implementação de medidas protecionistas.\cite{maskus2000quantifying}. Isso tem causado aparecimento de diversas barreiras não tarifárias (BNTs), que impedem o fluxo internacional de bens.
Apesar de estas barreiras poderem ser legítimas, por exemplo para corrigir eventuais externalidades negativas advindas do produto importado,  o fato é que estas terão impacto nas importações do país. Este impacto pode ser positivo ou negativo, dependendo do setor analisado. Em geral, normas de importação tendem a diminuir o comércio para bens primários e impulsionar o comércio de bens mais complexos \cite{moenius}.

\section{Metodologia}

Considerando os trabalhos que visam estabelecer quantitativamente o impacto das notificações, será utilizado um modelo de gravidade cujos estimadores serão estabelecidos por PPML (Poisson Pseudo Maximum Likelihood). 

\subsection{Modelos de gravidade}

Os modelos de gravidade são utilizados majoritariamente desde a década de 60 para a explicação de fluxos de comércio internacional. Originalmente derivado do modelo de Newton, utilizava a distância entre os dois objetos (países) e a massa deles (PIB), para explicar tal fluxo. Com o tempo, o desenvolvimento da área de economia internacional têm tornado o modelo cada vez mais teóricamente embasado e representativo da realidade. 

\subsection{Obtenção de estimadores}

Dada a característica de haver grandes quantidades de fluxos de troca com valor zero, o estimador utilizado será o de \emph{Poisson Pseudo Maximum Likelihood} (PPML). Isso se deve pelo fato de tal metodologia se portar melhor dados muitos valores nulos. Tal método de estimação também gera resultados consistentes na presença de heterocedasticidade. 

\subsection{Variáveis do modelo}

\begin{equation}
    \ln X_{JT} = \ln YB_t + \ln YE_{JT} + Dist_J + \ln TRF_{JT} 
\end{equation}

Onde:

\begin{itemize}
    \item $X_{JT}$ = Valor de exportação do Brasil para o país $J$ no período $T$;
    \item $YB_T$ = Renda do Brasil no período $t$;
    \item $YE_{JT}$ = Renda do país J no período $t$;
    \item $Dist_J$ = Distância entre o Brasil e o país $J$;
    \item $TRF_{JT}$ = Valor da tarifa efetivamente aplicada pelo país $J$ ao Brasil no período $T$.
\end{itemize}

Obs: Todas aquelas variáveis que não são dummies estão sendo transformadas por meio de logaritmo natural, assim como pede a especificação de \cite{Log_Of_Gravity}.


\section{Dados e fontes}

Bases de dados utilizadas:

\begin{enumerate}
    \item Notificações: https://www.epingalert.org/en;
    \item Valor de exportações: https://comtrade.un.org/;
    \item PIB : Banco mundial, \cite{WB};
    \item Distanciamento: \cite{CEPII};
    \item Tarifas: WITS
\end{enumerate}

\section{Resultados}

\subsection{Regressão Com base de dados CEPII}

Os resultados a seguir foram obtidos a partir da base de dados para modelos de gravidade da CEPII. Esta foi unificada com base própria criada a partir da quantidade de notificações de diferentes objetivos emitidas por diferentes países para diferentes commodities. Esta regressão não possui efeitos fixos, e não inclui variáveis tarifa.

% latex table generated in R 3.6.2 by xtable 1.8-4 package
% Mon Apr 13 11:24:30 2020
\begin{flushleft}
  
  \begin{table}[ht]
    \centering
    \begin{tabular}{l|l|l|l|l}
      \hline
      Variável & Coeficiente & Erro padrão & t valor & Pr($>$$|$t$|$) \\ 
      \hline
      (Intercepto) & 0.8289 & 0.4231 & 1.96 & 0.0501 \\ 
    dist\_log & -0.0023 & 0.0067 & -0.34 & 0.7362 \\ 
    Animal.health & 0.0041 & 0.0015 & 2.69 & 0.0072 \\ 
    Consumer.information & -0.0094 & 0.0110 & -0.86 & 0.3914 \\ 
    Food.safety & -0.0057 & 0.0009 & -6.64 & 0.0000 \\ 
    Harmonization & -0.5179 & 0.0285 & -18.19 & 0.0000 \\ 
    Lower.barriers.to.trade & -0.0332 & 0.0089 & -3.72 & 0.0002 \\ 
    Other & -0.0221 & 0.0483 & -0.46 & 0.6473 \\ 
    Plant.protection & 0.0009 & 0.0017 & 0.51 & 0.6101 \\ 
    PROT1 & -0.0523 & 0.0039 & -13.56 & 0.0000 \\ 
    PROT2 & 0.0037 & 0.0020 & 1.81 & 0.0708 \\ 
    PROT3 & -0.0366 & 0.0050 & -7.34 & 0.0000 \\ 
    PROT4 & -0.0073 & 0.0010 & -7.46 & 0.0000 \\ 
    PROT5 & -0.0851 & 0.0454 & -1.87 & 0.0611 \\ 
    PROT6 & 0.0274 & 0.0028 & 9.78 & 0.0000 \\ 
    Quality.requirements & -0.0158 & 0.0082 & -1.92 & 0.0553 \\ 
    ln\_gdp\_d & 0.0379 & 0.0014 & 27.62 & 0.0000 \\ 
    ln\_gdp\_o & 0.0328 & 0.0148 & 2.22 & 0.0263 \\ 
    comrelig & 0.0051 & 0.0107 & 0.48 & 0.6329 \\ 
    gatt\_d & -0.1269 & 0.0368 & -3.45 & 0.0006 \\ 
    eu\_d & -0.4327 & 0.0355 & -12.19 & 0.0000 \\ 
    \hline
  \end{tabular}
  \caption{Regresão 1}
\end{table}
\end{flushleft}

As seguintes variáveis foram encurtadas na tabela:
\begin{enumerate}
  \item PROT1: Prevention.of.deceptive.practices.and.consumer.protection;
  \item PROT2: Protect.humans.from.animal.plant.pest.or.disease;
  \item PROT3: Protect.territory.from.other.damage.from.pests;
  \item PROT4: Protection.of.Human.health.or.Safety;
  \item PROT5: Protection.of.animal.or.plant.life.or.health;
  \item PROT6: Protection.of.the.environment.
\end{enumerate}

\printbibliography

\end{document}